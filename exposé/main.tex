\documentclass[11pt,a4paper]{article}
\usepackage[utf8]{inputenc}
\usepackage{geometry}
\usepackage{booktabs}
\usepackage{parskip}
\usepackage{hyperref}
\usepackage{enumitem}
\usepackage{csquotes}
\usepackage[backend=biber,style=apa,natbib=true,sorting=nyt,maxbibnames=20]{biblatex}
\DeclareLanguageMapping{english}{english-apa}
\addbibresource{references.bib}

\geometry{
    a4paper,
    left=25mm,
    right=25mm,
    top=30mm,
    bottom=30mm,
}

\title{Comparing the Total Cost of Ownership (TCO) and Trust Properties of Off-Chain Computation Across Blockchains}
\author{Marcel Heidebrecht\\Information Systems Engineering, TU Berlin}
\date{\today}

\begin{document}

\maketitle

\begin{refsection}

\section*{1. Context}

Blockchain systems face a fundamental challenge: providing security and decentralization at scale while keeping transaction costs low. As global demand for decentralized applications grows, Layer-1 blockchains (like Ethereum and Solana) are evolving to offer various trade-offs between throughput, finality, and decentralization. To address scaling issues, many ecosystems adopt off-chain computation techniques (sometimes called Layer-2 or hybrid solutions), which allow complex operations to be executed outside the main chain and then verified or anchored on-chain. \parencite[pp.~1--2]{gorzny2024rollupcomparisonframework}

Typical examples include zk-rollups (using cryptographic proofs of correctness), optimistic rollups (using dispute periods), and emerging hybrid or modular approaches. Each Layer-1 ecosystem pairs with one or more Layer-2 options, creating a complex design space for developers and architects. \parencite[pp.~4--6]{cryptoeprint:2024/889}

For practitioners, understanding the economic and trust implications of choosing a given Layer-1 and Layer-2 pairing is critical. \parencite[p.~3]{cryptoeprint:2024/889} However, there is little systematic, practical guidance on the total cost of off-chain computation and the security-decentralization guarantees that come with it. \parencite[pp.~1--2]{gorzny2024rollupcomparisonframework}

\section*{2. Problem and Research Question}

Developers currently lack an objective way to compare the cost and qualitative properties of combining different off-chain computation models with different blockchains. Decisions are often based on fragmented information, popular trends, or partial cost analyses.

\textbf{Main Research Question:} How do total cost of ownership and trust properties of off-chain computation solutions differ when combined with different Layer-1 blockchains?

\textbf{Sub-questions:}
\begin{enumerate}[itemsep=0.5ex]
    \item What cost factors drive the TCO for off-chain computation across verification, data availability, and proof generation?
    \item How do these costs vary between major blockchains (e.g., Ethereum, Solana) and scaling models (e.g., zk-rollups, optimistic rollups)?
    \item What qualitative differences in security, decentralization, and finality are relevant for each platform and technology combination?
    \item How can these findings help guide architects and developers towards informed and application-specific choices?
\end{enumerate}

\section*{3. Solution Idea and Approach}

The thesis will use a combination of modeling, system prototyping, and empirical evaluation to compare mainstream off-chain computation approaches on multiple blockchains.

\begin{enumerate}[itemsep=0.5ex]
    \item Define a generic TCO model usable for any off-chain computation stack. Break down all economic costs, including off-chain computation (e.g., proof generation), on-chain costs (e.g., gas for verification or fraud proofs), and data availability (e.g., posting transaction data). 
    \item Select a representative set of Layer-1 blockchains and scaling solutions. Candidates include Ethereum, Solana, and their most common Layer-2 solutions such as zk-rollups and optimistic rollups. The study may use zk-rollups as a worked example given their strong privacy and integrity properties.
    \item Implement a small, standardized off-chain computation workload (e.g., simple balance update or threshold signature scheme) to test across all selected technologies.
    \item Deploy and test the workload on available testnets, using open-source tooling for proof generation and contract deployment. Collect all relevant cost and performance metrics.
    \item Evaluate and summarize qualitative aspects, including validator decentralization, time to finality, and known security risks, for each candidate system.
    \item Summarize results into a comparison table or matrix so architects can match workload and trust needs to one or more optimal deployment options.
\end{enumerate}

\section*{4. Aspired Implementation}

\begin{itemize}[itemsep=0.5ex]
    \item Create a reproducible off-chain computation workload for deployment on multiple blockchain ecosystems.
    \item Automate cost, performance, and resource usage measurements using scripts.
    \item Leverage available zk-rollup and rollup simulator libraries, as well as testnet RPC providers, to ensure realistic measurement.
    \item Produce all data and source code in an open, reproducible package suitable for future reference or extension.
\end{itemize}

\section*{5. Evaluation and Assessment}

\begin{itemize}[itemsep=0.5ex]
    \item Run case studies at different workload sizes (e.g., transaction batches of size 10, 100, and 1000).
    \item Measure TCO by breaking down all major costs for each Layer-1 and Layer-2 combination.
    \item Document runtime, proof generation, and on-chain costs.
    \item Compare security and liveness properties using external research, ecosystem trackers, and documentation.
    \item Present results in normalized tables and figures.
    \item Deliver a clear guide for practitioners summarizing the findings.
\end{itemize}

\section*{6. Timeline}

\begin{table}[ht]
\centering
\begin{tabular}{@{}lllp{6cm}@{}}
\toprule
\textbf{Phase} & \textbf{Start} & \textbf{End} & \textbf{Description} \\
\midrule
Research \& Selection        & Oct 13, 2025 & Nov 2, 2025   & Finalize scope, collect core readings \\
Prototype Implementation    & Nov 3, 2025  & Dec 7, 2025   & Develop generic workload and deploy tooling \\
Measurement Setup           & Dec 8, 2025  & Dec 28, 2025  & Automate experiments, run initial collection \\
Experiments \& Analysis     & Dec 29, 2025 & Jan 18, 2026  & Full tests, raw data analysis \\
Writing \& Revisions        & Jan 19, 2026 & Feb 3, 2026   & Results write-up and discussion \\
Buffer \& Final Submission  & Feb 4, 2026  & Mar 13, 2026  & Open for revisions, advisor review, submission \\
\bottomrule
\end{tabular}
\end{table}

\vfill
\noindent
\clearpage

\printbibliography[heading=subbibliography,title={References Cited in Exposé}]
\end{refsection}

\begin{refsection}
\nocite{gogol2025scalingdefizkrollups,cryptoeprint:2025/172,8525392,cryptoeprint:2019/360,ethereum_rollup_costs_2024,ethereum_optimistic_rollups_2025,Bappy_2025,wood_yellow_paper_2014,solana_whitepaper_2020,eip4844_2022,heinrich_tco_cloud_2023}
\printbibliography[heading=subbibliography,title={Preliminary Bibliography (For Later Thesis Stages)}]
\end{refsection}

\textbf{Contact:} marcel.heidebrecht@campus.tu-berlin.de

\textbf{GitHub:} \url{https://github.com/marcel-tu-berlin/master-thesis}
\end{document}
