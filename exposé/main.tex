\documentclass[11pt,a4paper]{article}

\usepackage[utf8]{inputenc}
\usepackage{geometry}
\usepackage{booktabs}
\usepackage{tabularx}
\usepackage{parskip}
\usepackage{hyperref}
\usepackage{enumitem}
\usepackage{csquotes}
\usepackage[backend=biber,style=apa,natbib=true,sorting=nyt,maxbibnames=20]{biblatex}

\DeclareLanguageMapping{english}{english-apa}
\addbibresource{references.bib}

\geometry{
a4paper,
left=25mm,
right=25mm,
top=30mm,
bottom=30mm
}

\title{Comparative Analysis of Trusted Execution Environments and zkVMs for Privacy-Preserving L1 Architectures}
\author{Marcel Heidebrecht\\Information Systems Engineering, TU Berlin}
\date{\today}

\begin{document}

\maketitle

\begin{refsection}

\section*{1. Context}

Blockchain systems face a fundamental trade-off between public verifiability and data privacy. While traditional distributed ledgers offer transparency, they are unsuitable for applications requiring confidentiality, such as enterprise supply chains or private financial settlements \parencite{eberhardt2018offchaining}. To address this, two distinct Layer-1 (L1) paradigms have emerged to enable privacy-preserving computation: hardware-based isolation via Trusted Execution Environments (TEEs) and cryptographic verification via Zero-Knowledge Virtual Machines (zkVMs).

\textbf{Oasis Network} represents the hardware-based approach, utilizing a modular architecture where consensus is separated from computation. Its ParaTimes leverage TEEs to execute smart contracts over encrypted data, ensuring that node operators cannot view the state they are computing. Oasis supports multiple TEE backends---including Intel SGX and TDX---each offering different trade-offs between Trusted Computing Base (TCB) size and ease of deployment. This architecture, originally introduced as Ekiden \parencite{cheng2019ekiden}, runs smart contracts in TEEs over confidential state while maintaining the scalability benefits of separating consensus from execution \parencite{oasis2025docs}. 

In contrast, \textbf{Aleo} represents the cryptographic approach, providing privacy guarantees for all transactions by default. Aleo utilizes the \textit{snarkVM} \parencite{provable2025snarkvm} and the \textit{Leo} \parencite{provable2025leo} programming language to execute state transitions off-chain and generate succinct validity proofs (zk-SNARKs) that are verified on-chain. This removes the need for trust in hardware manufacturers, relying instead on the hardness of discrete logarithm problems \parencite{aleo2025docs, ben2013snarkforc}.

The Trusted Compute Unit (TCU) framework \parencite{castillo2025tcu} suggests that these technologies share a functional goal---verifiable off-chain computation---but their architectural trade-offs differ significantly. While TEEs offer near-native performance, they require trust in hardware manufacturers and their attestation infrastructure. Conversely, zkVMs offer stronger trustlessness but historically suffer from high proof-generation latency \parencite{lavin2024zkps}.

\section*{2. Problem and Research Question}

Practitioners currently lack a rigorous decision framework for selecting between TEE-based L1s (Oasis) and zkVM-based L1s (Aleo) for specific application topologies. While informal comparisons exist in practitioner blogs and community articles, these typically lack systematic methodology and reproducible benchmarks. Consequently, technology selection is often driven by ideological preference rather than empirical metrics regarding cost, latency, and state management.

Specifically, the performance gap between these two approaches fluctuates based on the \textit{topology} of the workflow. TEEs generally excel at stateful, persistent applications (cyclic), while zkVMs are becoming increasingly efficient for stateless verification (linear). However, no peer-reviewed study has rigorously benchmarked these specific L1 protocols against identical logic flows.

\textbf{Research Question:} How do Oasis Network (TEE) and Aleo (zkVM) compare in terms of execution cost, finality latency, and trust properties when applied to linear versus cyclic workflow topologies?

\section*{3. Solution Idea and Approach}

This thesis will conduct a comparative evaluation by implementing two distinct workflow topologies---Linear and Cyclic---on both Oasis and Aleo.

\textbf{Technology Selection:}

\begin{itemize}[itemsep=0.3ex]

\item \textbf{TEE Platform (Oasis Network):} The \textit{Cipher ParaTime} will be used as the primary environment. It supports WebAssembly (WASM) smart contracts written in Rust, executing inside TEEs. Privacy is enforced through hardware isolation.

\item \textbf{zkVM Platform (Aleo):} The \textit{Leo} programming language and \textit{snarkVM} will be used. Logic will be compiled into arithmetic circuits, and execution will generate zero-knowledge proofs submitted to the Aleo Testnet. Privacy is enforced through cryptographic proofs.

\end{itemize}

\textbf{Workload Topologies:}

\begin{itemize}

\item \textbf{Linear Acyclic Graph (Supply Chain):} A unidirectional workflow where an asset passes through distinct verification steps (e.g., Producer $\rightarrow$ Logistics $\rightarrow$ Retailer).

\item \textbf{Cyclic Graph (Recurrent Financial State):} A stateful workflow where the output feeds back into the input (e.g., a private savings account updating interest).

\end{itemize}

\textbf{Methodology:}

The research will follow a design science approach involving the implementation of these topologies. To answer the research question, three evaluation dimensions will be examined:

\begin{itemize}
\item \textbf{Cost Efficiency:} Comparative cost (in native tokens/gas, normalized to USD) of executing a linear supply chain verification versus a cyclic compound interest calculation on Oasis Cipher versus Aleo.
\item \textbf{Performance \& Latency:} Wall-clock time from transaction submission to on-chain confirmation, comparing Aleo's proof generation and verification time against Oasis's execution-plus-attestation time for varying complexity levels.
\item \textbf{Trust Boundaries:} Qualitative analysis of security guarantees, comparing hardware-root-of-trust models with the cryptographic-root-of-trust model (zk-SNARKs on Aleo), particularly regarding upgradeability of the system and side-channel risks \parencite{munoz2023teesurvey}.
\item \textbf{Code Complexity:} Quantitative assessment using established software metrics: Lines of Code (LOC), Cyclomatic Complexity (CC) measuring independent control flow paths, and Halstead complexity measures (vocabulary size, program length, difficulty, and effort). These metrics will be compared for equivalent logic implementations in Rust (Oasis) vs. Leo (Aleo).
\end{itemize}

\section*{4. Aspired Implementation}

The implementation phase will involve developing a privacy benchmark suite deployed on both networks.

\textbf{Phase 1: Logic Specification}
A platform-agnostic specification of the Supply Chain and Savings Account algorithms will be defined to ensure logical equivalence.

\textbf{Phase 2: Aleo Implementation (Leo)}
The logic will be implemented in the Leo language. The focus will be on the \textit{Record} model---how Aleo handles state privacy by consuming and creating records. A local prover will be used to generate proofs for the Aleo Testnet.

\textbf{Phase 3: Oasis Implementation (Rust/WASM)}
The equivalent logic will be implemented as a smart contract for the Oasis Cipher ParaTime using the Oasis SDK. The implementation will explicitly utilize the confidential state store to demonstrate TEE capabilities.

\textbf{Phase 4: Benchmarking}
Scripts will automatically trigger transactions on both testnets.
\begin{itemize}
    \item \textit{Scenario A (Low Load):} Single transaction execution.
    \item \textit{Scenario B (High Load):} Batch processing to stress-test proof generation vs. enclave throughput.
\end{itemize}

\section*{5. Evaluation and Assessment}

The evaluation will synthesize quantitative metrics with qualitative architectural analysis.

\textbf{Quantitative Analysis:}
\begin{itemize}
\item \textbf{Latency Cross-over:} The hypothesis is that Aleo (zkVM) will have higher latency for complex cyclic computations due to proof generation overhead, whereas Oasis (TEE) will maintain linear scaling relative to execution complexity.
\item \textbf{Cost Structure:} Comparison of proof-based fee structures (Aleo) versus execution-based fee structures (Oasis).
\end{itemize}

\textbf{Qualitative Analysis:}
\begin{itemize}
\item \textbf{Security Model:} A critical comparison of hardware-based and cryptographic trust models. This includes discussing the implications of known TEE vulnerabilities (e.g., side-channel attacks) versus the risks of trusted setup ceremonies or circuit bugs in ZK systems \parencite{munoz2023teesurvey, tang2024zkpvulns}.
\item \textbf{Developer Experience:} Comparing the maturity of the Oasis Rust SDK vs. the Leo language ecosystem.
\end{itemize}

\section*{6. Scope Management}

To ensure feasibility, the following constraints are applied:

\begin{itemize}
\item \textbf{Excluded Features:} Cross-chain bridges and decentralized front-ends are excluded. The focus is strictly on the L1 protocol execution.
\item \textbf{Network Selection:} Only Oasis (Cipher) and Aleo are considered. Other privacy chains are out of scope.
\item \textbf{Testnet Only:} All deployments will occur on public testnets to avoid financial costs.
\end{itemize}

\section*{8. Timeline}

\begin{table}[ht]
\small
\begin{tabularx}{\textwidth}{@{}lcX@{}}
\toprule
\textbf{Phase} & \textbf{Weeks} & \textbf{Description} \\
\midrule
Literature Review \& Design & 1-3 & Study Oasis ParaTime architecture and Aleo SnarkVM/Leo docs. \\
Implementation (Aleo) & 4-6 & Develop Linear/Cyclic workflows in Leo. Deploy to Aleo Testnet. \\
Implementation (Oasis) & 7-9 & Develop workflows in Rust for Cipher ParaTime. Deploy to Oasis Testnet. \\
Benchmarking & 10-12 & Build \& Execute automated scripts. Collect gas and latency logs. \\
Evaluation \& Analysis & 13-15 & Analyze cost/latency trade-offs. Write Trust Boundary chapter. \\
Writing & 16-20 & Draft thesis chapters. \\
Final Polish & 21-24 & Final review and submission. \\
\bottomrule
\end{tabularx}
\end{table}

\printbibliography

\end{refsection}

\end{document}

